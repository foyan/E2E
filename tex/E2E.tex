% !TEX TS-program = pdflatex
% !TEX encoding = UTF-8 Unicode

% This is a simple template for a LaTeX document using the "article" class.
% See "book", "report", "letter" for other types of document.

\documentclass[11pt]{scrreprt} % use larger type; default would be 10pt

\usepackage[utf8]{inputenc} % set input encoding (not needed with XeLaTeX)
\usepackage{german}

%%% Examples of Article customizations
% These packages are optional, depending whether you want the features they provide.
% See the LaTeX Companion or other references for full information.

%%% PAGE DIMENSIONS
\usepackage{geometry} % to change the page dimensions
\geometry{a4paper} % or letterpaper (US) or a5paper or....
% \geometry{margin=2in} % for example, change the margins to 2 inches all round
% \geometry{landscape} % set up the page for landscape
%   read geometry.pdf for detailed page layout information

\usepackage{graphicx} % support the \includegraphics command and options
\usepackage{url}

% \usepackage[parfill]{parskip} % Activate to begin paragraphs with an empty line rather than an indent

%%% PACKAGES
\usepackage{booktabs} % for much better looking tables
\usepackage{array} % for better arrays (eg matrices) in maths
\usepackage{paralist} % very flexible & customisable lists (eg. enumerate/itemize, etc.)
\usepackage{verbatim} % adds environment for commenting out blocks of text & for better verbatim
\usepackage{subfig} % make it possible to include more than one captioned figure/table in a single float
% These packages are all incorporated in the memoir class to one degree or another...

%%% HEADERS & FOOTERS
\usepackage{fancyhdr} % This should be set AFTER setting up the page geometry
\pagestyle{fancy} % options: empty , plain , fancy
\renewcommand{\headrulewidth}{0pt} % customise the layout...
\lhead{}\chead{}\rhead{}
\lfoot{}\cfoot{\thepage}\rfoot{}

%%% SECTION TITLE APPEARANCE
\usepackage{sectsty}
\allsectionsfont{\sffamily\mdseries\upshape} % (See the fntguide.pdf for font help)
% (This matches ConTeXt defaults)

%%% ToC (table of contents) APPEARANCE
\usepackage[nottoc,notlof,notlot]{tocbibind} % Put the bibliography in the ToC
\usepackage[titles,subfigure]{tocloft} % Alter the style of the Table of Contents
\renewcommand{\cftsecfont}{\rmfamily\mdseries\upshape}
\renewcommand{\cftsecpagefont}{\rmfamily\mdseries\upshape} % No bold!

\usepackage{amsthm}

\theoremstyle{definition}
\newtheorem{definition}{Definition}


%%% END Article customizations

%%% The "real" document content comes below...

\title{IT Service Management: End-to-end-Überwachung}
\author{Florian Lüthi}
%\date{} % Activate to display a given date or no date (if empty),
         % otherwise the current date is printed 

\begin{document}
\maketitle

\tableofcontents

\chapter{Management Summary}

\chapter{Einführung}

\section{Kontext und Definitionen}

\begin{definition}[Business Service Management (BSM)]
Das Modell des Business Service Management verknüpft die Geschäftsprozesse eines Unternehmens mit den darunterliegenden IT-Services. Dadurch ist es möglich, die Abhängigkeiten von Business zu IT darzustellen, sowie die Auswirkungen von IT-Störungen auf das Business aufzuzeigen. Das Ziel von Business Service Management ist, eine bessere Abstimmung zwischen Business und IT zu erzielen. \cite{wiki:bsm}
\end{definition}

BSM startete ungefähr im Jahr 2003 als Buzzword, bis sich Forrester Research 2006 dem Thema annahm und begann, darüber zu publizieren. \cite{wiki:bsm, forrester:bsm, forrester:implementingBsm}

\begin{definition}[End-to-End-Monitoring]
Bla,.
\end{definition}

\section{Fragestellungen}

\subsection{Transparenz versus Transformabilität}

\subsection{Skalierbarkeit nach unten}

\chapter{Lösungsansätze}

\section{Neue Techniken}

\section{SLAs für E2E-Services}

Bla. \cite{EllisKauferstein200311}

\subsection{Metriken und KPIs}

Metriken... Blah... \cite{forrester:slaBestPractices}

\begin{enumerate}
\item Die Metrik ist objektiv messbar.
\item Die Metrik enthält eine klare Aussage über das erwartete Resultat.
\item Die Metrik unterstützt Business-seitige Anforderungen.
\item Die Metrik konzentriert sich entweder auf die Effektivität oder die Effizienz des zu messenden Prozesses.
\item Die Metrik erlaubt eine sinnvolle statistische oder Trend-Analyse.
\item Die Metrik wendet Industrie- und/oder andere Standards an.
\item Annahmen und Definitionen für zufriedenstellende Performance wird spezifiziert.
\item In die Festlegung der Metrik wurden diejenigen Stakeholder involviert, die nachher für die Performance des Prozesses verantwortlich sind.
\item Sowohl der Dienstleister wie auch der Abnehmer akzeptieren die Metrik.
\end{enumerate}

\subsection{WSLAs (Web Service Level Agreements)}

SOA (Service Oriented Architecture) ist ein aktuelles Buzzword: Innerhalb von Business-Prozessen agierende Applikationen, die sich dynamisch an ihre benötigten Services binden. Als Quasi-Standard haben sich in diesem Umfeld Web-Services durchgesetzt.

Die dafür notwendigen Definitionen sind mittlerweile ebenfalls standardisiert oder quasi-standardisiert:
\begin{description}
\item[SOAP] ein XML-basiertes Austauschformat für Messages \cite{wiki:soap}
\item[WSDL] ein XML-basiertes Austauschformat für Daten- und Operationen-Definitionen
\end{description}

Klassische SLAs sind als Freitext formuliert.

IBM Bla \cite{ibm:wslaSpec, ibm:wslaPaper}

\section{Gestaltung von IT-Leistungskatalogen mit E2E}

\section{Outsourcing: Auswirkungen von E2E}

Die Weiterentwicklung der Fähigkeiten der IT im Generellen hat viele neue Möglichkeiten gebracht, wie IT das Business direkt unterstützen und zur Wertschöpfungskette beitragen kann. Gemäss Forrester (\cite{forrester:slaBestPractices}) zeitigen sich daraus allerdings zwei sich widersprechende Effekte:

\begin{description}
\item[Die IT-Abhängigkeit des Business nimmt zu.] Praktisch sämtliche Unternehmen haben einen Punkt erreicht, an dem ihre Business-Prozesse so abhängig von einer funktionierenden IT sind wie von Elektrizät und fliessendem Wasser.
\item[Der Kostendruck auf die IT nimmt zu.] Externe Dienstleister haben für die Mehrheit der IT-Geschäftsfelder Lösungen entwickelt, die direkt mit den Dienstleistungen der internen IT-Abteilungen konkurrenzieren können. Gemäss der Logik des Marktes resultieren daraus sinkende Preise und steigende Qualität.
\end{description}

Die interne IT kann sich nun entweder dieses Konkurrenz- und Kostendrucks annehmen, sich auf den Einkauf von externen Services und deren Weitervermittlung an das Business konzentrieren, oder eine Mischform der ersten beiden anstreben.

Alle drei Strategien setzen aber voraus, dass die IT ihre operationelle Struktur derer eines IT-Dienstleisters angleicht. Dafür braucht sie entsprechendes Qualitätsmanagement, um dem Business ihren Wert beweisen zu können.

\chapter{Fazit}

\bibliography{Referenzen}
\bibliographystyle{plain}

\end{document}
